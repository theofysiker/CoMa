\documentclass{llncs}

\usepackage{graphicx} % to be able to include graphics
\usepackage[ngerman]{babel}

\begin{document}

\pagestyle{headings}               % switches on printing of running heads


\mainmatter                        % start of the contributions

\title{Computerorientierte Mathematik I}
\subtitle{\"Ubung 0}
\titlerunning{Computerorientierte Mathematik I\\
\"Ubung 0}

\author{Samantha Albrecht-Scharmacher\inst{1} (BSC)\\Nicolas Lehmann\inst{2} (Dipl. Kfm., BSC)}
\authorrunning{Samantha Albrecht-Scharmacher \& Nicolas Lehmann} % abbreviated author list (for running head)
\tocauthor{Samantha Albrecht-Scharmacher, Nicolas Lehmann}

\date{\today}

\institute{
Freie Universit\"at Berlin, FB Mathematik und Informatik,\\
Institut f\"ur Informatik, \email{scharbrecht@zedat.fu-berlin.de}
\and
Freie Universit\"at Berlin, FB Mathematik und Informatik,\\Institut f\"ur Informatik, AG Datenbanksysteme, Raum 170,\\
\email{mail@nicolaslehmann.de}, \texttt{http://www.nicolaslehmann.de}
}

\maketitle

\begin{center}
\includegraphics{fubsiegel.jpg}
\end{center}

\chapter*{L\"osungen zu den gestellten Aufgaben}

\section*{Aufgabe 1}

\underline{Inhalte}:
\begin{itemize}
\item[2.] \textbf{Der Umgang mit Unix}\\(unwichtig, falls eigene MATLAB Entwicklungsumgebung vorhanden)
\begin{itemize}
\item Einloggen / Ausloggen
\item Dateisystem
\item Programme starten und beenden
\item Prozessemanagement
\end{itemize}
\item[3.] \textbf{Eine Beispielsitzung}\\(absolute MATLAB Basics, unwichtig, falls MATLAB bereits bekannt)
\begin{itemize}
\item Matrizen
\item Operationen
\item flops / tics /tocs
\end{itemize}
\item[4.] \textbf{Bedingte Verzweigungen und Schleifen}\\(absolute Programmier-Basics, unwichtig, falls MATLAB bereits bekannt)
\begin{itemize}
\item if
\item for
\item while
\item break
\end{itemize}
\end{itemize}


\section*{Aufgabe 2}

\subsection*{Teilaufgabe a)}

...einloggen...

\subsection*{Teilaufgabe b)}

...Terminal starten...

\subsection*{Teilaufgabe c)}

...MATLAB starten...

\subsection*{Teilaufgabe d)}

\begin{verbatim}
ans = 56
a = 7
b = 8
ans = -1
ans = 0.8750
c = 15
c = -1
c = 56
ans = 8
a = 7
\end{verbatim}


\subsection*{Teilaufgabe e)}

\begin{verbatim}
x =

     1     2     3


ans =

     1
     2
     3
\end{verbatim}


\subsection*{Teilaufgabe f)}

\begin{verbatim}
Undefined function or variable 'y'.

Error in Uebung0 (line 29)
y
\end{verbatim}


\subsection*{Teilaufgabe g)}

\begin{verbatim}
cos - Cosine of argument in radians

    This MATLAB function returns the cosine for each element of X.

    Y = cos(X)

    Reference page for cos

    See also acos, acosd, cosd, cosh

    Other uses of cos
        symbolic/cos

sin - Sine of argument in radians

    This MATLAB function returns the sine of the elements of X.

    Y = sin(X)

    Reference page for sin

    See also asin, asind, sind, sinh

    Other uses of sin
        symbolic/sin
\end{verbatim}

\subsection*{Teilaufgabe h)}

Hilfefenster \"offnet sich...


\subsection*{Teilaufgabe i)}

\begin{verbatim}
Logarithmus not found.
\end{verbatim}


\section*{Aufgabe 3}

\subsection*{Teilaufgabe a)}

...neue MATLAB Datei \"offnen...

\subsection*{Teilaufgabe b)}

\begin{verbatim}
A = [];
for i=1:3:9
   v = colon(i,i+2);
   A = vertcat(A,v); 
end

% oder A = [1 2 3; 4 5 6; 7 8 9]

x = ones(1,3);
% oder x = [1 1 1]

A % Ausgabe
x % Ausgabe
\end{verbatim}


\subsection*{Teilaufgabe c)}

\begin{verbatim}

\end{verbatim}


\subsection*{Teilaufgabe d)}

\begin{verbatim}

\end{verbatim}

\end{document}
