\documentclass{llncs}

\usepackage{graphicx} % to be able to include graphics
\usepackage[ngerman]{babel}
\usepackage{amsmath}
\usepackage{amssymb}
\usepackage{stmaryrd}

\begin{document}

\pagestyle{headings}               % switches on printing of running heads


\mainmatter                        % start of the contributions

\title{Computerorientierte Mathematik I}
\subtitle{\"Ubung 2}
\titlerunning{Computerorientierte Mathematik I\\
\"Ubung 2}

\author{Gideon Schr\"oder\inst{1}\\Samanta Scharmacher\inst{2}\\Nicolas Lehmann\inst{3} (Dipl. Kfm., BSC)}
\authorrunning{Samanta Scharmacher \& Nicolas Lehmann \& Gideon Schr\"oder} % abbreviated author list (for running head)
\tocauthor{Samanta Scharmacher, Nicolas Lehmann, Gideon Schr\"oder}

\date{\today}

\institute{
Freie Universit\"at Berlin, FB Physik,\\
Institut f\"ur Physik, \email{gideon.2610@hotmail.de}
\and
Freie Universit\"at Berlin, FB Mathematik und Informatik,\\
Institut f\"ur Informatik, \email{scharbrecht@zedat.fu-berlin.de}
\and
Freie Universit\"at Berlin, FB Mathematik und Informatik,\\Institut f\"ur Informatik, AG Datenbanksysteme, Raum 170,\\
\email{mail@nicolaslehmann.de}, \texttt{http://www.nicolaslehmann.de}
}

\maketitle

\begin{center}
\includegraphics{fubsiegel.jpg}
\end{center}

\chapter*{L\"osungen zu den gestellten Aufgaben}

\section*{Aufgabe 1}

\subsection*{Teilaufgabe a)}
\subsection*{Teilaufgabe b)}
\subsection*{Teilaufgabe c)}

\section*{Aufgabe 2}
\subsection*{Teilaufgabe a)}
\subsection*{Teilaufgabe b)}


\section*{Aufgabe 3}

\subsection*{Teilaufgabe a)}

%https://www.informatik.uni-leipzig.de/~meiler/MuP.dir/MuPWS11.dir/Vorlesung/Kap00_MZ.pdf (Seite 5)
Zu zeigen: 
\begin{center}
\textit{Jeder endliche Dualbruch ist auch ein endlicher Dezimalbruch.}
\end{center}
Ein beliebiger Dualbruch ist darstellbar als:
\begin{align*}
\sum_{i=-m}^{n} z_{i} \cdot 2^{i} &= \sum_{i=-m}^{-1} z_{i} \cdot 2^{i} + \sum_{i=0}^{n} z_{i} \cdot 2^{i} \\
                                  &= \sum_{i=1}^{m} z_{-i} \cdot 2^{-i} + \sum_{i=0}^{n} z_{i} \cdot 2^{i} \\
                                  &= \sum_{i=0}^{m-1} z_{1-i} \cdot 2^{1-i} + \sum_{i=0}^{n} z_{i} \cdot 2^{i}\\
\Leftrightarrow \sum_{i=-m}^{n} z_{i} \cdot \left( \frac{10}{5} \right)^{i} &= \sum_{i=-m}^{n} z_{i} \cdot \left(\frac{1}{5}\right)^{-i} 10^{i} \\
                                  &= \left(\frac{1}{5}\right)^{m-n} \sum_{i=-m}^{n} z_{i} \cdot 10^{i} \\
                                  &= \left(\frac{1}{5}\right)^{m-n} \sum_{i=-m}^{-1} z_{i} \cdot 10^{i} + \sum_{i=0}^{n} z_{i} \cdot 10^{i} \\
                                  &= \left(\frac{1}{5}\right)^{m-n} \sum_{i=1}^{m} z_{-i} \cdot 10^{-i} + \sum_{i=0}^{n} z_{i} \cdot 10^{i} \\
                                  &= \left(\frac{1}{5}\right)^{m-n} \sum_{i=0}^{m-1} z_{1-i} \cdot 10^{1-i} + \sum_{i=0}^{n} z_{i} \cdot 10^{i}
\end{align*}
$\hfill \square$

% = \sum_{i=1}^{m} z_{i} \cdot \left( \frac{10}{5} \right)^{-i} + \sum_{i=0}^{n} z_{i} \cdot \left( \frac{10}{5} \right)^{i} \\
%\Leftrightarrow  &= \left(\frac{1}{5}\right)^{m+1} \sum_{i=-m}^{-1} z_{i} \cdot 10^{i} + \sum_{i=0}^{n} z_{i} \cdot \left( \frac{10}{5} \right)^{i} = \sum_{i=1}^{m} z_{i} \cdot \left( \frac{10}{5} \right)^{-i} + \sum_{i=0}^{n} z_{i} \cdot \left( \frac{10}{5} \right)^{i}


\subsection*{Teilaufgabe b)}

Angenommen es gilt:
\begin{center}
\textit{Jeder endliche Dezimalbruch ist auch ein endlicher Dualbruch.}
\end{center}
Dann w\"are die Dezimalzahl $0,4$ als endlicher Dualbruch darstellbar.
\begin{center}
$0,4_{10} = 0,\overline{0110}_2$, Widerspruch $\lightning$
\end{center}

\end{document}
