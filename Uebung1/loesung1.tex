\documentclass{llncs}

\usepackage{graphicx} % to be able to include graphics
\usepackage[ngerman]{babel}

\begin{document}

\pagestyle{headings}               % switches on printing of running heads


\mainmatter                        % start of the contributions

\title{Computerorientierte Mathematik I}
\subtitle{\"Ubung 1}
\titlerunning{Computerorientierte Mathematik I\\
\"Ubung 0}

\author{Samanta Scharmacher\inst{1}\\Nicolas Lehmann\inst{2} (Dipl. Kfm., BSC)}
\authorrunning{Samanta Scharmacher \& Nicolas Lehmann} % abbreviated author list (for running head)
\tocauthor{Samanta Scharmacher, Nicolas Lehmann}

\date{\today}

\institute{
Freie Universit\"at Berlin, FB Mathematik und Informatik,\\
Institut f\"ur Informatik, \email{scharbrecht@zedat.fu-berlin.de}
\and
Freie Universit\"at Berlin, FB Mathematik und Informatik,\\Institut f\"ur Informatik, AG Datenbanksysteme, Raum 170,\\
\email{mail@nicolaslehmann.de}, \texttt{http://www.nicolaslehmann.de}
}

\maketitle

\begin{center}
\includegraphics{fubsiegel.jpg}
\end{center}

\chapter*{L\"osungen zu den gestellten Aufgaben}

\section*{Aufgabe 1}

\subsection*{Teilaufgabe a)}

$55_{10} = 5 \cdot 10^1 + 5 \cdot 10^0 = 55_{10} = 1 \cdot 7^2 + 0 \cdot 7^1 + 6 \cdot 7^0 = 106_{7}$

\subsection*{Teilaufgabe b)}

$42_{7} = 4 \cdot 7^1 + 2 \cdot 7^0 = 30_{10} = 1 \cdot 3^3 + 0 \cdot 3^2 + 1 \cdot 3^1 + 0 \cdot 3^0 = 1010_{3}$

\subsection*{Teilaufgabe c)}

$12321_{4} = 1 \cdot 4^4 + 2 \cdot 4^3 + 3 \cdot 4^2 + 2 \cdot 4^1 +1 \cdot 4^0 = 441_{10} = 1 \cdot 2^8 + 1 \cdot 2^7 + 0 \cdot 2^6 + 1 \cdot 2^5 + 1 \cdot 2^4 + 1 \cdot 2^3 + 0 \cdot 2^2 + 0 \cdot 2^1 + 1 \cdot 2^0 = 110111001_{2}$

\subsection*{Teilaufgabe d)}

$17HAI_{26} = 1 \cdot 26^4 + 7 \cdot 26^3 + H \cdot 26^2 + A \cdot 26^1 + I \cdot 26^0 = 592454_{10} = C \cdot 36^3 + 25 \cdot 36^2 + 5 \cdot 36^1 + 2 \cdot 36^0 = C52_{36}$

\section*{Aufgabe 2}

\section*{Aufgabe 3}

... siehe \texttt{dual1.m} und \texttt{dual2.m}

\section*{Aufgabe 4}

\subsection*{Teilaufgabe a)}

\begin{verbatim}
     15 = 01111
     -5 = 11011 <-- 11010 <-- 00101
15+(-5) = 01111 + 11011 = 01010
     10 = 01010
\end{verbatim}

\begin{verbatim}
     15 = 1111
      5 =  101
   15-5 = 1111 - 101 = 1010
     10 = 1010
\end{verbatim}


\subsection*{Teilaufgabe b)}

Rechnung f\"ur $3+(-2)$
\begin{verbatim}
-------------- Zweierkomplement
     3 = 00011
    -2 = 11110 <-- 11101 <-- 00010
3+(-2) = 00011 + 11011 = 00001
     1 = 00001
     
-------------- Einerkomplement
     3 = 00011
    -2 = 11101 <-- 00010
3+(-2) = 00011 + 11101 = 00000
     0 = 00000
\end{verbatim}
Rechnung f\"ur $3+(-3)$
\begin{verbatim}
-------------- Zweierkomplement
     3 = 00011
    -3 = 11101 <-- 11100 <-- 00011
3+(-3) = 00011 + 11101 = 00000
     0 = 00000

-------------- Einerkomplement
     3 = 00011
    -3 = 11100 <-- 00011
3+(-3) = 00011 + 11100 = 11111
    -1 = 11111
\end{verbatim}
Rechnung f\"ur $3+(-4)$
\begin{verbatim}
-------------- Zweierkomplement
     3 = 00011
    -4 = 11100 <-- 11011 <-- 00100
3+(-4) = 00011 + 11100 = 11111
    -1 = 11111
    
-------------- Einerkomplement
     3 = 00011
    -4 = 11011 <-- 00100
3+(-4) = 00011 + 11011 = 11110
    -2 = 11110
\end{verbatim}

\subsection*{Teilaufgabe c)}

Die m\"ogliche Range einer 5-Bit Zweierkomplementzahl wird \"uberschritten.\\
$$minBound(Zweierkomplement_{5 Bit}) = 2^{4}$$
$$maxBound(Zweierkomplement_{5 Bit}) = 2^{4} -1$$

\end{document}
