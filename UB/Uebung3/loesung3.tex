\documentclass{llncs}

\usepackage{graphicx} % to be able to include graphics
\usepackage[ngerman]{babel}
\usepackage{amsmath}
\usepackage{amssymb}
\usepackage{stmaryrd}

\begin{document}

\pagestyle{headings}               % switches on printing of running heads


\mainmatter                        % start of the contributions

\title{Computerorientierte Mathematik I}
\subtitle{\"Ubung 2}
\titlerunning{Computerorientierte Mathematik I\\
\"Ubung 2}

\author{Gideon Schr\"oder\inst{1}\\Samanta Scharmacher\inst{2}\\Nicolas Lehmann\inst{3} (Dipl. Kfm., BSC)}
\authorrunning{Samanta Scharmacher \& Nicolas Lehmann \& Gideon Schr\"oder} % abbreviated author list (for running head)
\tocauthor{Samanta Scharmacher, Nicolas Lehmann, Gideon Schr\"oder}

\date{\today}

\institute{
Freie Universit\"at Berlin, FB Physik,\\
Institut f\"ur Physik, \email{gideon.2610@hotmail.de}
\and
Freie Universit\"at Berlin, FB Mathematik und Informatik,\\
Institut f\"ur Informatik, \email{scharbrecht@zedat.fu-berlin.de}
\and
Freie Universit\"at Berlin, FB Mathematik und Informatik,\\Institut f\"ur Informatik, AG Datenbanksysteme, Raum 170,\\
\email{mail@nicolaslehmann.de}, \texttt{http://www.nicolaslehmann.de}
}

\maketitle

\begin{center}
\includegraphics{fubsiegel.jpg}
\end{center}

\chapter*{L\"osungen zu den gestellten Aufgaben}

\section*{Aufgabe 1}

F\"ur die durchgef\"uhrte Rechnung gilt:
\begin{align*}
                X &= \left( \begin{matrix} 1 & 181 & 40\\1 & 175 & 65\\1 & 180 & 50\\1 & 170 & 25\\1 & 178 & 48 \\ 1 & 182 & 52\\1 & 185 & 36\\ 1 & 170 & 60\end{matrix} \right) \\
(X' \cdot X)^{-1} &= \left( \begin{matrix} 154.8644 & -0.8504 & -0.0786 \\ -0.8504 & 0.0047 & 0.0002 \\ -0.0786 & 0.0002 & 0.0009 \end{matrix} \right) \\
(X' \cdot X)^{-1} &\neq \left( \begin{matrix}154.86 & -0.85 & -0.08\\-0.85 & 0.005 & 0.0002\\-0.08 & 0.0002 & 0.0008\end{matrix} \right)
\end{align*}
De Frage \glqq Wer hat Recht?\grqq kann nicht eindeutig beantwortet werden.

\section*{Aufgabe 2}

\subsection*{Teilaufgabe a)}

\subsection*{Teilaufgabe b)}

\section*{Aufgabe 3}

\subsection*{Teilaufgabe a)}

\subsection*{Teilaufgabe b)}
\end{document}
